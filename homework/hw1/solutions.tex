%%%%     SETTING STARTS - DO NOT CHANGE Unless your TeX setting require so   %%
%%%%%%%%%%%%%%%%%%%%%%%%%%%%%%%%%%%%%%%%%%%%%%%%%%%%%%%%%%%%%%%%%%%%%%%%%%%%%%%
%%----------------------------------------------------------------------------------
% DO NOT Change this. It is the required setting A4 page, 11pt, onside print, book style
%%----------------------------------------------------------------------------------
\documentclass[a4paper,11pt,oneside]{book}

%%-------------------------------------
%% Page margin settings - % half inch margin all sides (recommended)
%%-------------------------------------
\usepackage[margin=1.2in]{geometry} 

%%-------------------------------------
%% Font settings - % CM San or Ariel (recommended)
%%-------------------------------------
% Switch the following two line off: to revert back to default LaTex font (NOT recommended)
\usepackage{amsfonts}
\renewcommand*\familydefault{\sfdefault}

%%-------------------------------------
%% Math/Definition/Theorem/Algorithm packages settings 
%%-------------------------------------
\usepackage[cmex10]{amsmath}
\usepackage{amssymb}
\usepackage{amsthm}
\usepackage{fancyhdr}
\usepackage{amsmath}
\newtheorem{mydef}{Definition}
\newtheorem{mytherm}{Theorem}
%%-------------------------------------
%% Algorithms/Code Listing environment settings  - 
%% Please do not change these settings
%%-------------------------------------
\usepackage{algorithm}
\usepackage{algpseudocode}
\renewcommand{\algorithmicrequire}{\textbf{Input:}}
\renewcommand{\algorithmicensure}{\textbf{Output:}}
\usepackage[utf8]{inputenc}
\usepackage{listings}
\usepackage{xcolor}
\definecolor{codegreen}{rgb}{0,0.6,0.1}
\definecolor{codegray}{rgb}{0.5,0.5,0.5}
\definecolor{codeblue}{rgb}{0.10,0.00,1.00}
\definecolor{codepurple}{rgb}{0.58,0,0.82}
\definecolor{backcolour}{rgb}{1.0,1.0,1.0}

\lstdefinestyle{mystyle}{
    backgroundcolor=\color{backcolour},   
    commentstyle=\color{codegreen},
    keywordstyle=\color{codeblue},
    numberstyle=\tiny\color{codegray},
    stringstyle=\color{codepurple},
    basicstyle=\ttfamily\footnotesize,
    breakatwhitespace=false,         
    breaklines=true,                 
    captionpos=b,                        
    keepspaces=true,                 
    numbers=left,                    
    numbersep=5pt,                  
    showspaces=false,                
    showstringspaces=false,
    showtabs=false,                  
    tabsize=2,
    frame=none
}
\lstset{style=mystyle}

%%-------------------------------------
%% Graphics/Figures environment settings
%%-------------------------------------
\usepackage{graphicx}
\usepackage{subfigure}
\usepackage{caption}
\usepackage{lipsum}

%%-------------------------------------
%% Table environment settings
%%-------------------------------------
\usepackage{multirow}
\usepackage{rotating}
\usepackage{makecell}
\usepackage{booktabs}
%\usepackage{longtable,booktabs}

%%-------------------------------------
%% List of Abbreviations settings
%%-------------------------------------
\usepackage{enumitem}
\newlist{abbrv}{itemize}{1}
\setlist[abbrv,1]{label=,labelwidth=1in,align=parleft,itemsep=0.1\baselineskip,leftmargin=!}

%%-------------------------------------
%% Bibliography/References settings   - Harvard Style was used in this report
%%-------------------------------------
\usepackage[hidelinks]{hyperref}
\usepackage[comma,authoryear]{natbib}
\renewcommand{\bibname}{References} % DO NOT remove or switch of 

%%-------------------------------------
%% Appendix settings     
%%-------------------------------------
\usepackage[toc]{appendix}
%%%%%%%%%%%%%%%%%%%%%%%%%%%%%%%%%%%%%%%%%%%%%%%%%%%%%%%%%%%%%%%%%%%%%%%%%%%%%%%%%%%%%%%
%%%%                     SETTING ENDS                                            %%%%%%
%%%%%%%%%%%%%%%%%%%%%%%%%%%%%%%%%%%%%%%%%%%%%%%%%%%%%%%%%%%%%%%%%%%%%%%%%%%%%%%%%%%%%%%


\fancyhf{}
\fancyhead[L]{EE4033-Algorithms Spring 2023}
\fancyhead[R]{Steven Wong, T11705207}
\renewcommand\headrulewidth{0pt}
\pagestyle{fancy}

\begin{document}
\noindent\makebox[\textwidth][c]{\Large\bfseries Homework Assignment 1}
\normalsize

\begin{enumerate}
  \item {\textbf{Exercise 2.2$-$2}} 
  \\ The loop invariant is that for any iteration $i$, the smallest $i-1$ elements
  will be sorted in ascending order. Thus, we only need to run the algorithm on the first $n-1$
  elements because the smallest $n-1$ elements will be sorted at that point. In other words, the $n$th 
  remaining number must be the greatest in our array. In both best and worst cases, the running time of the
  algorithm is $\theta(n^2)$.
\begin{algorithm}
    \caption{Selection Sort Algorithm}
    \begin{algorithmic}[1]
        \Require Unsorted Array $\mathbf{x}$
        \Ensure Sorted Ascending Order Array $\mathbf{x}$
        \Statex
        \Function{SelectionSort}{$\mathbf{x}$}
          \For{$i \gets 0$ to length(x)$-1$}
          \State{$smallest \gets x[i]$}
          \For{$j \gets i+1$ to length(x)}
            \If{$x[j] < smallest$}
            \State {$smallest \gets x[j]$}
            \EndIf
          \EndFor
          % Swap the smallest element with the current element
          \State {$x[j] \gets x[i]$}
          \State {$x[i] \gets smallest$}
        \EndFor
        \EndFunction
    \end{algorithmic}
\end{algorithm}

  \item {\textbf{Exercise 2.3$-$3}}
  \\ \underline{Base case}: For $n = 2$, we have $nlogn = 2\log2 = 2 \cdot 1 = 2$. Therefore, $T(n)$ is true for $n = 2$.
  \\
  \\ \underline{Inductive Step}: Assume that $T(n/2)$ holds for $T(n/2) = (n/2)\log(n/2)$ then, we must show that this holds for $T(n)$.
  \\
  \begin{align*}
    T(n) &= 2T(n/2) + n\\
    &= 2(n/2)\log(n/2) + n\\
    &= n(\log(n)-1) + n\\
    &= n\log(n)
  \end{align*}
  \\ \underline{Therefore}, $T(n)$ is true for all $n \geq 2$ which are exact powers of 2.
  
  \item {\textbf{Problem 2$-$3}} 
  \\ a. The running time of this code fragment is $\Theta(n)$.
  \\ b. The running time of the naive polynomial evaluation algorithm is $\Theta(n^2)$.
  \begin{algorithm}
    \caption{Naive Polynomial-Evaluation Algorthm Part b}
    \begin{algorithmic}[1]
        \Statex
        \Function{NaiveEval}{$\mathbf{A, x}$}
        \State z = $0$
        \For {$i \gets 1$ to $A.length$}
        \State coeff = $1$
        \For {$j \gets 1$ to $i-1$}
          \State coeff *= x
        \EndFor
        \State z += $A[i] \cdot \text{coeff}$
        \EndFor
        \\
        \Return z
      \EndFunction
    \end{algorithmic}
\end{algorithm} 
  \\ c.
  \\ d. This is trivial to verify because Homer's algorithm returns the correct value for any polynomial.
  
  \item {\textbf{Prove or disprove $f(n) + g(n) = \Theta(\text{max}(f(n), g(n)))$}}
  \\ With the assumption that $f(n)$ and $g(n)$ are non-negative functions, $\max(f(n), g(n))$ produces either 
  $f(n)$ or $g(n)$ for all $n \geq0$. This means that $f(n) + g(n) \leq \max(f(n), g(n))$ for all $n \geq 0$. Therefore, $f(n) + g(n) = \Theta(\text{max}(f(n), g(n)))$.

  \item {\textbf{Problem 3.3a}} 
  \\ The expressions can be ranked into the following order: 
  $
    2^{2^{n+1}} \rightarrow 2^{2^n} \rightarrow (n+1)! \rightarrow e^n \rightarrow n!
    \rightarrow n \cdot 2^n
    \rightarrow 2^n
    \rightarrow (3 / 2)^n
    \rightarrow (\log n) !
    \rightarrow n^3
    \rightarrow n^2 
    \rightarrow 4^{\log n}
    \rightarrow n \log n
    \rightarrow \log (n !)
    \rightarrow n=2^{\log n}
    \rightarrow (\sqrt{2})^{\log n}
    \rightarrow 2^{\sqrt{2 \log n}}
    \rightarrow \log ^2 n
    \rightarrow \ln n
    \rightarrow \ln \ln n 
    \rightarrow \sqrt{\log n}
    \rightarrow 2^{\log n}
    \rightarrow \log { }^* n
    \rightarrow \log ^*(\log n)
  $

  \item {\textbf{Exercise 4.1$-$5}}
  \\
  \begin{algorithm}
    \caption{Maximum Subarray Problem}
    \begin{algorithmic}[1]
        \Statex \textbf{Input :} A non-empty array $\mathbf{A}$ of $n$ numbers.
        \Function{MaxSubArray}{$\mathbf{A}$}
        \State $maxSoFar \gets A[0]$
        \State $maxSumHere \gets A[0]$
        \State $startIndex \gets 0$
        \For {$i \gets 1$ to $A.length$}
        % If maxSumHere is positive, add A[i] to it
        % Otherwise, set maxSumHere to A[i] and update startingIndex to i
        \If {$maxSumHere > 0$}
          \State $maxSumHere \gets maxSumHere + A[i]$
        \Else
          \State $maxSumHere \gets A[i]$
          \State $startIndex \gets i$
        \EndIf
        % If maxSumHere is greater than maxSoFar, update maxSoFar
        \If {$maxSumHere > maxSoFar$}
          \State $maxSoFar \gets maxSumHere$
        \EndIf
        \EndFor
        \\
        \Return $\mathbf{(maxSoFar)}$
        \EndFunction
    \end{algorithmic}
\end{algorithm}
\\


  \item {\textbf{Exercise 4.2$-$4}}
  \\ Assuming that we can multiply $3 \times 3$ matrices in $k$ multiplications, then for an $n \cdot n$ matrix, using $n/3$ matrices in 
  $T(n) =  kT(n/3) + O(n^2)$. Using the master method, if case 2 applies we have $T(n) = O(n^2logn)$ and $k = 9$.
  If case 1 applies, we have $T(n) = O(n^{\log_3 k})$ and $k = 21$. Thus, $k = 21$ is the largest matrix we can multiply in
  time $O\left(n^{\log 7}\right)$.

  \item {\textbf{Exercise 4.3$-$7}}
  \\ Using the master method in Section 4.5, you can show that the solution to the recurrence $T(n)=4 T(n / 3)+n$ is $T(n)=\Theta\left(n^{\log _3 4}\right)$. Show that a substitution proof with the assumption $T(n) \leq c n^{\log _3 4}$ fails. Then show how to subtract off a lower-order term to make a substitution proof work.

  
  \item {\textbf{Exercise 4.4$-$9}}
   Use a recursion tree to give an asymptotically tight solution to the recurrence $T(n)=T(\alpha n)+T((1-\alpha) n)+c n$, where $\alpha$ is a constant in the range $0<\alpha<1$ and $c>0$ is also a constant.

  \item {\textbf{Problem 4-3bfhj}} 
    
\end{enumerate}
\end{document}